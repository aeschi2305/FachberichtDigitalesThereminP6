\subsection{Cordic Algorithmus}\label{subsec:Cordic}

Um in einem FPGA aufwendigere Rechenoperationen wie die Berechnung eines Sinus zu implementieren ist eine zusätzliche Hardware notwendig. Der Cordic Algorithmus kann für diesen Zweck eingesetzt werden. Nebst anderen diversen Rechenoperationen ist der Einsatz als Sinusgenerator möglich, was in späteren Kapiteln genauer besprochen ist. \\
Der Cordic Algorithmus ist ein iterativer Algorithmus welcher praktisch nur Additionen und Verschiebungen von Bits benötigt. 
Er kann in zwei Modi betrieben werden. Zum einen der Vektor Modus, in welchem die Berechnung eines Winkels aus einem gegebenen Vektor möglich ist. Zum anderen der Rotationsmodus, mit welchem aus einem gegebenen Winkel die Elemente des zugehörigen Vektors berechnet werden können. Die folgenden Formeln sind für diese Berechnung notwendig \cite{Cordic}:

\begin{equation}
x_{i+1} = x_i - y_id_i2^{-i}
\label{equ:cordic_1}
\end{equation} 
\begin{equation}
y_{i+1} = y_i + x_id_i2^{-i}
\label{equ:cordic_2}
\end{equation} 
\begin{equation}
z_{i+1} = z_i - d_i\arctan{2^{-i}}
\label{equ:cordic_3}
\end{equation} 

Dabei ist die Berechnung von \(d_i\) im Rotationsmodus wie folgt: 

\begin{equation}
d_i=
\begin{cases}
-1 &z_i < 0 \\
1 &\text{otherwise}
\end{cases}
\label{equ:cordic_4}
\end{equation} 

Formeln \ref{equ:cordic_1} bis \ref{equ:cordic_3} zeigen schön den iterativen Ablauf des Algorithmus auf. Um nun einen Sinuswert zu berechnen sind folgende Initialwerte notwendig:

\begin{equation}
\begin{aligned}
x_0 = 1 \\
y_0 = 0 \\
z_0 = \varphi
\end{aligned}
\label{equ:cordic_3}
\end{equation} 

\(\varphi\) ist der gegebene Winkel, welcher zwischen \(-\pi/2\) und \(\pi/2\) sein muss, damit der Algorithmus konvergiert.

Daraus ergeben sich nach \(n\) Iterationen der Sinus und Kosinus Wert wie folgt:

\begin{equation}
\begin{aligned}
x_n = \frac{\cos{\varphi}}{A} \\
y_n = \frac{\sin{\varphi}}{A}
\end{aligned}
\label{equ:cordic_3}
\end{equation} 

Schlussendlich ist es notwendig die Resultate um den Faktor \(A = 0.60725294\) zu korrigieren um die richtigen Werte zu erhalten.

