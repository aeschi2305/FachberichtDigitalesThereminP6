\subsection{Cordic Algorithmus}\label{subsec:Cordic}

Um in einem FPGA aufwendigere Rechenoperationen wie die Berechnung eines Sinus zu implementieren ist eine zusätzliche Hardware notwendig. Weit verbreitet ist dafür der Cordic Algorithmus. Nebst anderen diversen Rechenoperationen ist der Einsatz als Sinusgenerator möglich, was in späteren Kapiteln genauer besprochen ist. \\
Der Cordic Algorithmus ist ein iterativer Algorithmus, welcher praktisch nur Additionen und Verschiebungen von Bits benötigt. 
Der Algorithmus besitzt zwei Modi. Zum einen der Vektor Modus, in welchem die Berechnung eines Winkels aus einem gegebenen Vektor möglich ist. Zum Anderen der Rotationsmodus, mit welchem die Berechnung der Elemente eines Vektors aus einem gegebenen Winkel möglich ist. Die folgenden Formeln sind für diese Berechnung notwendig \cite{Cordic}:

\begin{equation}
x_{i+1} = x_i - y_id_i2^{-i}
\label{equ:cordic_1}
\end{equation} 
\begin{equation}
y_{i+1} = y_i + x_id_i2^{-i}
\label{equ:cordic_2}
\end{equation} 
\begin{equation}
z_{i+1} = z_i - d_i\arctan{2^{-i}}
\label{equ:cordic_3}
\end{equation} 

\(x_i\) und \(y_i\) sind dabei die Elemente des Vektors und \(z_i\) ist der Winkel des Vektors. \(x_{i+1}\),\(y_{i+1}\) und \(z_{i+1}\) sind die Resultate einer Iteration.
Die Berechnung von \(d_i\) im Rotationsmodus lautet wie folgt: 

\begin{equation}
d_i=
\begin{cases}
-1 &z_i < 0 \\
1 &\text{otherwise}
\end{cases}
\label{equ:cordic_4}
\end{equation} 

Formeln \ref{equ:cordic_1} bis \ref{equ:cordic_3} zeigen schön den iterativen Ablauf des Algorithmus auf. Um nun einen Sinuswert zu berechnen sind folgende Initialwerte notwendig:

\begin{equation}
\begin{aligned}
x_0 = 1 \\
y_0 = 0 \\
z_0 = \varphi
\end{aligned}
\label{equ:cordic_5}
\end{equation} 

\(\varphi\) ist der gegebene Winkel, welcher zwischen \(-\pi/2\) und \(\pi/2\) sein muss, damit der Algorithmus konvergiert.

Daraus ergeben sich nach \(n\) Iterationen der Sinus und Kosinus Wert wie folgt:

\begin{equation}
\begin{aligned}
x_n = \frac{\cos{\varphi}}{A} \\
y_n = \frac{\sin{\varphi}}{A}
\end{aligned}
\label{equ:cordic_6}
\end{equation} 

Schlussendlich ist es notwendig die Resultate um den Faktor \(A = 0.60725294\) zu korrigieren um die richtigen Werte zu erhalten.

