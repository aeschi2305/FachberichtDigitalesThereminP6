\subsection{Antennenoszillator PCB}\label{subsec:PCB}
Auf dem Antennenoszillator PCB haben wir die Betriebsspannungen und die Ausgänge der Komperatoren getestet. Die dazu verwendeten Messmittel sind in Tabelle \ref{tab:Gemessene_Spannungen_PCB} angegeben. Die Signale haben wir mit dem Lecroy Wavesurfer 3054 Oszilloskop gemessen. 

Die \SI{9}{V} Betriebsspannung hat eine Rippelspannung von \SI{7.3}{mV}. Somit ist die Colpitts Oszillator-Schaltung stabil gespiesen. Die \SI{3.3}{V} Betriebsspannung weist eine Rippelspannung von \SI{133}{mV} auf. Dies ist jedoch noch vertretbar, da die Spannung für den digitalen Teil der Schaltung gebraucht wird.
\begin{table}[H]
	\centering
	\caption{Gemessene Spannungen PCB}
	\label{tab:Gemessene_Spannungen_PCB}
	\begin{tabular}{l|l|l|l}
		\textbf{Spannung} & \textbf{soll [VDC]} & \textbf{ist [VDC]} &	\textbf{Ripple [mVDC]}\\
		\hline \hline
		
		Betriebsspannung 9V & 9 & 9.3 &  7.3 \\ 
		&      &   &   \\ 
		\hline
		Betriebsspannung 3.3V & 3.3 & 3.4 &  133 \\ 
		&     &     &   \\ 
		\hline
		Schaltnetzteil 12V & 12 & 12.5 &  150 \\ 
		&     &       &   \\ 
		\hline
		
	\end{tabular}
\end{table} 

Bei den Messungen der Ausgänge der Komparatoren waren auf den Rechtecken bei der steigenden und fallenden Flanke Überschwinger ersichtlich. In einem Paper von Analog Device haben wir erfahren, dass die Verursacher der Überschwinger die langen Ausgangsleitungen sind. Diese wirken wie nicht abgeschlossene Übertragungsleitungen und lösen Reflexionen aus \cite{comparator_techniques}. Um dieses Problem zu lösen, schlossen wir die Ausgangsleitung mit einem \SI{300}{Ohm} Widerstand ab.

Beim ersten Austesten der Oszillatoren mit angeschlossenen Antennen wiesen die Oszillatoren ähnliche Frequenzen auf. Bei einer Veränderung der Lautstärkenoszillatorfrequenz wurde ebenfalls die Tonhöhenantennenfrequenz verändert. Dies kann durch Übersprechen oder gegenseitige Beeinflussung der Oszillatoren hervorgerufen werden. Um dieses Problem zu umgehen, haben wir die Frequenzen der beiden Oszillatoren um \SI{30}{kHz} versetzt zu gewählt. Dadurch war das Problem behoben. 

Durch den häufigen Gebrauch des PCB ist uns bewusst geworden, dass der verwendete JFet sehr empfindlich gegenüber elektrostatischer Entladung ist.

Bei einer Weiterentwicklung des Theremins sollte das Antennenoszillator PCB überarbeitet werden. Es ist eine Schutzbeschaltung für den JFet notwendig. Zudem sollten die Oszillatoren auf zwei separaten PCB realisiert werden, um Übersprechen und gegenseitige Beeinflussung zu vermeiden. 

\begin{table}[H]
	\centering
	\caption{Verwendete Messmittel}
	\label{tab:Verwendete_Messmittel}
	\begin{tabular}{l|l}
		\textbf{Messgerät} & \textbf{Bezeichnung}	\\
		\hline \hline
		
		Lecroy wavesurfer 3054  &  MSZ-M-0091  \\ 
		&        \\ 
		\hline
		
	\end{tabular}
\end{table} 