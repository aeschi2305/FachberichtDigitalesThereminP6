\clearpage
\section{Schlusswort}\label{sec:Schlusswort}
Im Rahmen des Projekt 5  wurde eine digitale Plattform für die Verarbeitung von Signalen einer Thereminantenne entwickelt. Alle Komponenten ausser der Antennenoszillator wurden in VHDL realisiert. Die VHDL Komponenten wurden so realisiert, dass diese im Projekt 6 weiter gebraucht werden können.
Momentan lässt sich das Theremin ohne Lautstärkeantenne spielen. Über zwei Taster kann der digitale Referenzoszillator manuell auf die Frequenz des Tonhöhenoszillator abgestimmt werden. Sobald das Theremin kalibriert ist kann es Töne von ca. 100-2000Hz spielen. 
Die Ziele welche in der Projektklärung definiert wurden konnten erfüllt werden. Bei der kontinuierlichen Tongenerierung gibt es noch eine Unschönheit bei der Ansteuerung des Codec. Es ist im generierten Ton ein Knacken zu hören, welches auf einen Fehler in der Ansteuerung des verwendeten Codec zurückzuführen ist. Dieser Fehler besteht nach wie vor. Jedoch wird diese Ansteuerung in Projekt 6 sowieso anders realisiert.

Im Projekt 6 wird die zweite Antenne implementiert, um gleichzeitig die Lautstärke einstellen zu können. Des weiteren soll es möglich sein diskrete Töne zu spielen. Dieser Modus soll es Anfängern ermöglichen bekannte Melodien nachspielen zu können. \\
Damit das theoretische Wissen aus dem Fach digitale Schaltungstechnik (dst) in die Praxis umgesetzt werden kann, wird ein Nios Soft Core Prozessor implementiert. Dieser übernimmt die Ansteuerung des Codec und die Modus Verwaltung. Beim starten des Theremin soll ein automatisches Tuning des Referenzoszillators stattfinden. Dazu wird der digitale Referenzoszillator auf die Frequenz des Antennenoszillator abgestimmt. Um das Theremin für Messen zu verwenden wird das DE1-SoC Board und die Antennenoszillatoren mit den Antennen in ein ansprechendes Gehäuse verbaut. Die Antennen sollen abgeschraubt werden können um einen komfortablen Transport zu ermöglichen.

Als erstes wird im Projekt 6 mit der Implementierung der Lautstärkeantenne auf dem FPGA und dem Redesign des PCB begonnen. Zudem muss Recherche in das Thema Nios Soft Core Prozessor angestellt werden, um diesen später implementieren zu können. 






