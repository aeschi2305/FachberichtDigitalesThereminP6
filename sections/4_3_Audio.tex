\subsection{Audio}\label{subsec:audio}

Für die Kommunikation mit WM8731 Codec verwenden wir die IP Componente Audi\_Video Configuration von Intel. Diese besitzt die Funktion 
\textit{alt\_up\_av\_config\_write\_audio\_cfg\_register} um die Kontrollregister des Codec über den Hardware Abstraction Layer (HAL) anzusprechen. \todo{Quelle Intel hinzufügen}.
Das Module \textbf{audio.c} beinhaltet die beiden Funktionen, \textit{codec\_wm8731\_init()} und \textit{set\_vol(vol\_gain)}. 

In der Initialisierung wird der Codec als Master konfiguriert \todo{Verweiss auf Dennis}. Der Clock des Codec wird auf 12MHz gesetzt und die Sampling Rate auf 48kHz. Die Input Audio Data Bit Länge wird auf 24 Bit gesetzt und der Übertragungsmodus der Daten auf left justified gesetzt. Mit diesen Einstellungen ist eine Kominikation mit der audio serializer Komponente möglich \todo{Verweiss auf Dennis}.
Um Störungen zu vermeiden sind die Line In Eingänge des Codec gemutet, da wir nur den Line Out brauchen. Der Linke und Rechte Kanal sind so eingestellt das beide Kanäle die selben Lautstärken haben. 

Die Gesamtlautstärke des Theremin kann in 10 unterschiedliche Pegel eingestellt werden, dies geschieht mit dem Aufruf der Funktion \textit{set\_vol(vol\_gain)}. Der Audio Signal Pegel wird auf der leisesten Stufe um -76db unterdrückt. Für jede Stufe grösser wird der Pegel um 7db weniger unterdrückt. Mit dem höchsten Pegel wird das Audio Signal um -7db gedämpft. Der Codec könnte gemäss Datenblatt das Signal bis +6dB verstärken. Nach Labor Tests haben wir uns entschieden das Audio Signal nur zu dämpfen, da eine Verstärkung zu laut war.