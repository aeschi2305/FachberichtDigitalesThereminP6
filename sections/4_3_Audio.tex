\subsection{Audio}\label{subsec:audio}

Die Kommunikation mit WM8731 Codec geschieht über die IP Komponente Audio/Video Configuration Core von Intel. Diese besitzt die Funktion 
\textit{alt\_up\_av\_config\_write\_audio\_cfg\_register}, um die Kontrollregister des Cores über die Hardware Abstraction Layer (HAL) anzusprechen \cite{Audio_config}.
Das Modul \textbf{audio.c} beinhaltet die beiden Funktionen, \textit{codec\_wm8731\_init()} und\\ \textit{set\_vol(vol\_gain)}. 

In der Initialisierung wird der Codec als Master konfiguriert. Der Clock des Codec wird auf \SI{12}{MHz} gesetzt und die Sampling Rate auf \SI{48}{kHz}. Die Input Audio Data Bit Länge wird auf \SI{24}{Bit} und der Übertragungsmodus der Daten auf Left Justified gesetzt. Mit diesen Einstellungen ist eine Kommunikation mit der Audioserialisierer-Komponente möglich.
Um Störungen zu vermeiden, sind die Line In Eingänge des Codec stumm geschaltet, da nur der Line Out nötig ist. Der linke und rechte Kanal sind so eingestellt, dass beide Kanäle die selben Lautstärken haben \cite{codec}. 

Die Gesamtlautstärke des Theremins ist auf 10 unterschiedliche Pegel einbestellbar. Dies geschieht mit dem Aufruf der Funktion \textit{set\_vol(vol\_gain)}. Die leiseste Stufe dämpft den Pegel um \SI{76}{db}. Jeweils eine Stufe grösser reduziert die Dämpfung um \SI{7}{db}. Die höchste Stufe dämpft den Pegel um \SI{7}{db}. Der Codec könnte gemäss Datenblatt das Signal bis +6dB verstärken. Nach Labor-Tests haben wir uns entschieden, das Audio Signal nur zu dämpfen, da eine Verstärkung zu laut ist und Rauschen verursacht.

Ursprünglich war geplant, die Lautstärke des Theremins über den Codec zu steuern. Nicht wie in Kapitel \ref{subsec:Volume_Generation} beschrieben über die VHDL Komponente. Diese Methode konnte jedoch nicht realisiert werden, da die Zero-Cross Detection des Codec nicht funktionierte, obwohl das entsprechende Flag gesetzt war. Mit der Zero-Cross Dedection sollte der Codec erst bei einem Nulldurchgang die Lautstärke vermindern oder erhöhen. In unseren Versuchen hat sich die Lautstärke jedoch nie geändert, obwohl ein Sinus Audio Signal angelegt war. Wir mussten feststellen, dass diese Methode für den DAC nicht funktioniert. Mit dieser Methode hätte die Lautstärke, verglichen zur realisierten Methode, eine bessere Dynamik gehabt. 