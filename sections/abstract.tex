\begin{abstract}
% Description aim/objective
\noindent In this Project a Theremin was built that mainly operates on digital hardware unlike the original device that solely used analog electronics. The device is supposed to be used in presentations for trade fairs by the Institute for Sensors and Electronics ISE. As such the device should be built in a appealing housing. Moreover the device should have other additional functionality such as soundeffects or the ability to record sound.
% Method
The digital hardware was implemented in VHDL on the developer board DE1-SoC from terasIC with a Cyclone V FPGA from Intel. The sole analog component implemented was the oscillator that controls the pitch. 
% Results
The pitch of the device can be changed well, but the sound itself has a flaw at the moment, because there is an audible crackle. This is due to a communication problem with the codec that was used for the audio output. This problem will not be corrected during this project, because the communication will be implemented differently in the finished product.
% Conclusion
The work in this project served as a platform for the continuation in project 6. The next steps would be to implement the volume control and redesign the pcb for two antennas and oscillators.


\end{abstract}	
