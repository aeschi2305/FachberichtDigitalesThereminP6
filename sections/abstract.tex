
\begin{abstract}
% Description aim/objective
\noindent This project is a continuation on the work on a Theremin that mainly operates on digital hardware. Unlike the original device, which solely used analog electronics. The device is supposed to be used in presentations for trade fairs by the Institute for Sensors and Electronics ISE. As such the device should be built in an appealing housing. Moreover the device should have other additional functionality that helps the player to use the instrument more easily.
% Method
The digital hardware was implemented in VHDL on the developer board DE1-SoC from Terasic with a Cyclone V FPGA from Intel. The sole analog component implemented were the oscillators that control the pitch and volume by use of two antennae and the codec, wich converts the audio data for use on a loudspeaker. 
% Results
The pitch and volume of the instrument can be changed by the two antennae like a usual theremin. To adjust settings the theremin has a touchscreen display that is controlled by the Nios system, that was implemented. The theremin has two functionalities, that help the player use the instrument more easily. First the glissando effect, wich corrects the pitch to the closest musical note. Second the pitch display, wich displays the closest musical note the player ist playing at and the deviation that the player is causing. It has two minor flaws. At higher frequencies aliasing is audible. And with certain volume settings a bit of noise is perceptible.
% Conclusion
In conclusion it can be said, that the theremin is ready for any trade fairs that will come. 


\end{abstract}	
