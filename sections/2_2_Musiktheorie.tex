\subsection{Musiktheorie}\label{subsec:Musiktheorie}

Um besser an einem Musikinstrument arbeiten zu können ist es wichtig ein wenig Musiktheorie zu kennen. Der wichtigste Fakt ist, dass unser Gehör den Schallpegel logarithmisch wahrnimmt und in Dezibel (dB) angegeben wird. Auch die Frequenz der Tonhöhe hören wir nicht linear. Ein Ton mit \SI{400}{Hz} nehmen wir nicht als doppelt so hoch wahr als ein Ton mit \SI{200}{Hz}. Dies ist sehr schön ersichtlich in Tabelle \ref{tab:Toene_Frequenzen}. Je höher die Töne ansteigen, desto höhere Frequenzunterschiede entstehen zwischen ihnen. Für einfacheres Rechnen von diesen Unterschieden ist die Masseinheit Cent gebräuchlich. Dabei ist definiert, dass zwei Töne \SI{100}{Cent} auseinanderliegen und dass zwei Töne mit einer Oktave Unterschied \SI{1200}{Cent} Frequenzunterschied haben. Diese Cent-Werte kann man mithilfe von Formel \ref{equ:Cent} in einen Faktor umrechnen \cite{Cent}.

\begin{equation}
x = \sqrt[\leftroot{-2}\uproot{1}1200]{2}^{n_{cent}}
\label{equ:Cent}
\end{equation} 

Dabei ist \(n_{cent}\) der Unterschied in Cent und \(x\) als Faktor.\\ Ist nun die ``akustische'' Mitte zwischen zwei Tönen gesucht ist die Berechnung mit Cent nützlich. Diese Mitte liegt nicht linear zwischen den beiden Tönen sondern \SI{50}{Cent} entfernt von beiden Tönen. Werden diese \SI{50}{Cent} in einen Faktor umgerechnet und mit dem tieferen Ton multipliziert erhält man diese ``akustische'' Mitte.

Um nun zu sagen, wann ein Unterschied in der Frequenz vom Gehör wahrgenommen wird, kommt es ganz auf die Person drauf an. Als Faustformel kann gesagt werden, dass das Gehör zwei aufeinanderfolgende Töne mit etwa \SI{6}{Cent} unterschied registriert. Jedoch ist es schwierig zu sagen wann das Gehör einen Ton als ''nicht getroffen`` empfindet.\cite{Cent}

Ein weiteres interessantes Thema ist die Pentatonik. Dabei handelt es sich um ein Tonsystem mit nur 5 Tönen. Ein gutes Beispiel dafür sind die schwarzen Tasten des Klavier. Benützt der Spieler nur diese Tasten, spielt er in einem pentatonischen Tonsystem. In Abbildung \ref{tab:Toene_Frequenzen} entspricht dies allen Tönen mit einem \# in der Notation. Ein Merkmal der Pentatonik ist, dass es sehr einfach ist eine Melodie zu spielen, die ansprechend klingt, ohne grossen Aufwand.\cite{Pentatonik}



\begin{table}[H]
	\centering
	\caption{Töne aus vier Oktaven und deren Frequenzen \cite{Toene_Frequenzen}}
	\label{tab:Toene_Frequenzen}
	\begin{tabular}{l|l|l|l|l|l}
		\textbf{Ton} & \textbf{Frequenz[Hz]} & \textbf{Ton} & \textbf{Frequenz[Hz]} &\textbf{Ton} & \textbf{Frequenz[Hz]} \\
		\hline\hline
		C3 	& 130.813 	& F4	 & 349.228		& A\#5	 &  932.328	 \\ \hline
		C\#3 & 138.591 	& F\#4	 & 369.994		& B5	 &  987.767	 \\ \hline
		D3 	& 146.832 	& G4	 & 391.995		& C6	 &  1046.5	 \\ \hline
		D\#3 & 155.563 	& G\#4	 & 415.305		& C\#6	 &  1108.73	 \\ \hline
		E3 	& 164.814 	& A4	 & 440		 	& D6	 &  1174.66	 \\ \hline
		F3 	& 174.614 	& A\#4	 & 466.164		& D\#6	 &  1244.51	 \\ \hline
		F\#3 & 184.997 	& B4	 & 493.883		& E6	 &  1318.51	 \\ \hline
		G3 	& 195.998 	& C5	 & 523.251		& F6	 &  1396.91	 \\ \hline
		G\#3 & 207.652 	& C\#5	 & 554.365		& F\#6	 &  1479.98	 \\ \hline
		A3 	& 220 		& D5	 & 587.33		& G6	 &  1567.98	 \\ \hline
		A\#3 & 233.082 	& D\#5	 & 622.254		& G\#6	 &  1661.22	 \\ \hline
		B3 	& 246.942 	& E5	 & 659.255		& A6	 &  1760	 \\ \hline
		C4 	& 261.626 	& F5	 & 698.456		& A\#6	 &  1864.66	 \\ \hline
		C\#4 & 277.183 	& F\#5	 & 739.989		& B6	 &  1975.53	 \\ \hline
		D4 	& 293.665 	& G5	 & 783.991		& C7	 &  2093	 \\ \hline
		D\#4 & 311.127 	& G\#5	 & 830.609		&  		 &  		 \\ \hline
		E4 	& 329.628 	& A5	 & 880		 	&  		 &  		 \\ \hline
	
		
	\end{tabular}
\end{table}