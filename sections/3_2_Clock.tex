\subsection{Clock}\label{subsec:Clock}
Die verschiednen Clocks für die Hardwarekomponenten und die CPU werden in zwei PLL Blöcken generiert. Ein Block für die Signalverarbeitung und einer für das Nios System. In Tabelle \ref{tab:clocks} sind alle Frequenzen aufgelistet. \\
Alle Frequenzen welche nicht 50MHz sind ergaben sich daraus, dass die externen Peripherien, welche mit den entsprechenden IP Cores verbunden sind, diese Frequenzen als Vorgabe haben. Weiter benötigen die Komponenten Pitch- und Volume Generation die Frequenz 54Mhz, da deren Frequenz ein vielfaches von 48kHz sein muss.

\begin{table}[H]
	\centering
	\caption{Clockfrequenzen der verschiedenen Komponenten}
	\label{tab:clocks}
	\begin{tabular}{l|l|l}
		\textbf{Komponente} & \textbf{Frequenz} & \textbf{PLL Core} \\
		\hline\hline
		Nios Processor & \SI{50}{MHz} & PLL CPU  \\ \hline
		JTAG Controller & \SI{50}{MHz} & PLL CPU \\ \hline
		Timer & \SI{50}{MHz} & PLL CPU \\ \hline
		SysID & \SI{50}{MHz} & PLL CPU \\ \hline
		DRAM Controller & \SI{50}{MHz} & PLL CPU \\ \hline
		SDRAM & \SI{50}{MHz} & PLL CPU \\ \hline
		LCD Controller & \SI{15}{MHz} & PLL CPU \\ \hline
		LCD Reset & \SI{15}{MHz} & PLL CPU \\ \hline
		Touch Interrupt & \SI{15}{MHz} & PLL CPU \\ \hline
		Touch Busy & \SI{15}{MHz} & PLL CPU \\ \hline
		Touch SPI & \SI{15}{MHz} & PLL CPU \\ \hline
		Audio Config & \SI{12}{MHz} & PLL CPU \\ \hline
		Flash Controller & \SI{25}{MHz} & PLL CPU \\ \hline
		Pitch Generation & \SI{54}{MHz} & PLL Signal-Processing \\ \hline
		Volume Generation & \SI{54}{MHz} & PLL Signal-Processing \\ \hline
		DC-FIFO Input	& \SI{54}{MHz} & PLL Signal-Processing \\ \hline
		DC-FIFO Output	& \SI{24}{MHz} & PLL Signal-Processing \\ \hline
	 	Audio Serializer	& \SI{24}{MHz} & PLL Signal-Processing \\ \hline
	 	Codec	& \SI{12}{MHz} & PLL Signal-Processing \\ \hline
		
	\end{tabular}
\end{table}
