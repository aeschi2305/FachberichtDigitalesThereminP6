\subsection{Gesamttest}\label{subsec:Gesammttest}
Zuletzt haben wir das den gesamten Aufbau auf dessen Funktionsfähigkeit getestet. Dabei sind drei Punkte aufgefallen.\\
Erstens tritt das leichte Aliasing, welches wir erwartet haben bei hoher Lautstärke auf. Auffallend ist, dass dieses nur bei den höheren Tönen (über \SI{1}{kHz}) auftritt. Dies ist auf die Filterung des Mischersignals durch die CIC-Filter und das verwenden eines Rechtecksignals im Mischer zurückzuführen. Die Oberwellen des Rechtecksignal welche auch Mischprodukte hervorrufen, machen es schwierig durch Filterung mittels CIC-Filter Aliasing ganz zu vermeiden. Diese Problematik ist in Kapitel \ref{subsec:CIC_Filter} beschrieben.\\
Zweitens ist bei einer bestimmten Einstellung des Theremin ein Rauschen zu hören. Stellt man die Lautstärke über das Display auf das Maximum und über die Lautstärkenantenne eine tiefe Lautstärke ein, ist ein leichtes Rauschen zu hören. Dies ist auf den Kompromiss, auf welchen wir in Kapitel \ref{subsec:audio} eingegangen sind, zurückzuführen. Da der Digital-Analog-Wandler im Codec in diesem Fall ein nicht voll ausgesteuertes digitales Signal erhält, fällt dessen Rauschen mehr ins Gewicht.\\
Weiter ist uns erst nach der festen


