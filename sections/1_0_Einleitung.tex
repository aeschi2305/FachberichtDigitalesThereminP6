\clearpage
\section{Einleitung}\label{sec:Einleitung}
Das Theremin kennen heutzutage nur wenige Leute, obwohl es das erste elektronische Instrument war. Es wurde 1920 von dem Russen Lev Sergejewitsch Termen, welcher sich später zu Leon Theremin umbenennen liess, erfunden \cite{Theremin_h}. Personen die regelmässig Filme schauen, haben die Musik welche mit einem Theremin gemacht wird bestimmt schon einmal gehört. Ein Beispiel dafür ist Ghostbusters, wo das Theremin oft im Hintergrund zu hören ist. Zudem ist das Theremin in einigen Science-Fiction-Filmen und Horrorfilmen zu hören, wegen seines unheimlichen Klangs \cite{Goast_m}.\\
Das Theremin wird ohne es zu berühren gespielt, indem der Spieler mit den Händen die Distanz zu zwei Antennen ändert. Dies führt zu Veränderung der Tonhöhe mit der einen Antenne und der Lautstärke mit der anderen Antenne. Die Antennen und der Spieler wirken dabei wie Kondensatoren, welche zwei Oszillatoren ganz leicht verstimmen. Da diese Oszillatoren mit etwa \SI{500}{kHz} schwingen machen diese kleinen Änderungen eine Frequenzänderung im Bereich von mehreren kHz aus. Weitere Verarbeitung bringt diese Änderungen in den hörbaren Bereich. \\
Dass das Theremin über kontinuierliche Veränderungen der Tonhöhe gespielt wird ist eine typische Eigenschaft des Instruments.

Im Projekt 5 wurden die Grundlagen für ein solches Instrument entwickelt. Die Implementation des sonst analogen Instruments ist nun jedoch digital. Dabei ist es auf einem Field Programmable Gate Array (FPGA) implementiert. Das Theremin soll als Messeobjekt für das Institut für Sensorik und Elektronik ISE dienen, um die Möglichkeiten welche FPGAs bieten aufzuzeigen. Nach Ende des Projekt 5 war die Implementation, welche das Spielen über die Tonhöhenantenne ermöglicht weitestgehend abgeschlossen. Der Antennenoszillator war der einzige analoge Teil des Gerätes. Die Realisierung der restlichen Komponenten war in VHDL. Das Resultat wurde auf dem DE1-SoC Board von Terasic mit einem Cyclone V FPGA von Intel getestet. Quartus Prime diente als Entwicklungsumgebung.

Das Projekt 6 umfasste die abschliessende Entwicklung des digitalen Theremin. Die Lautstärke kann nun über eine zweite Antenne eingestellt werden. Es ist nun somit möglich das digitale Theremin wie ein normales Theremin zu spielen.\\
Weiter ist eine Bedienung des Theremin über ein LCD mit Touchscreen möglich. Der implementierte Nios II Prozessor ist dabei für diese Bedienung und die Steuerung der restlichen Komponenten zuständig.\\
Über das Display kann der Spieler eine Kalibration des Theremin durchführen, welche automatisch die Tonhöhe und Lautstärke richtig einstellt.\\
Für Spieler mit etwas weniger musikalischem Talent können zwei Spielhilfen aktiviert werden. Einerseits der Glissando-Effekt und andererseits die Anzeige der Spielgenauigkeit mit zwei verschiedenen Tonleitern. Der Glissando-Effekt hilft dem Spieler, indem er während dem Spielen auf den nächstgelegenen Ton korrigiert. Die Anzeige der Spielgenauigkeit zeigt dem Spieler wie genau der Ton getroffen ist und bei gewissen Einstellungen, wie nahe er an dem richtigen Ton ist.
Das Gehäuse ist als 3D-Druck entstanden, um ein ausgefalleneres Design zu ermöglichen.


Die folgende Dokumentation beginnt mit den Technischen Grundlagen, wo als erstes beschrieben ist, wie ein Theremin funktioniert und anschliessend diverse Grundlagen welche für die Implementation nötig waren. Als nächstes folgen Kapitel zu der Realisierung der Hardware, Software und dem Gehäuse. Abschliessend folgt die Validierung des Theremin und eine Dokumentation der Ergebnisse in einem Schlusswort.






