\clearpage
\section{Einleitung}\label{sec:Einleitung}
Das Theremin kennen heutzutage nur wenige Leute, obwohl es das erste elektronische Instrument war. Es wurde 1920 von dem Russen Lev Sergejewitsch Termen, welcher sich später zu Leon Theremin umbenennen liess, erfunden \cite{Theremin_h}. Personen die regelmässig Filme schauen, haben die Musik welche mit einem Theremin gemacht wird bestimmt schon einmal gehört. Ein Beispiel dafür ist Ghostbusters, wo das Theremin oft im Hintergrund zu hören ist. Zudem ist das Theremin in einigen Science-Fiction-Filmen und Horrorfilmen zu hören \cite{Goast_m}. Das Theremin wird ohne es zu berühren gespielt, indem man mit den Händen die Distanz zu zwei Antennen ändert. Dies führt zu Veränderung der Tonhöhe und Lautstärke.

Im Projekt 5 und 6 soll nun ein solches Instrument entwickelt werden. Mit dem Unterschied, dass das sonst analoge Instrument digital aufgebaut werden soll. Dabei soll es auf einem Field Programmable Gate Array (FPGA) implementiert werden. Später soll das Theremin als Messeobjekt für das Institut für Sensorik und Elektronik ISE verwendet werden. Im Rahmen des Projekt 5 wurde die Tonhöhenantenne des Theremin realisiert. Dazu wurde die Antenne zusammen mit dem Antennenoszillator analog beibehalten. Die restlichen Komponenten wurden in VHDL realisiert. Das Resultat wurde auf dem DE1-SoC Board von terasIC mit einem Cyclone V FPGA von Intel getestet.

Der folgende Fachbericht beginnt mit dem Kapitel \ref{sec:Technische Grundlagen} Technische Grundlagen.In der ersten Hälfte des Kapitel wird erklärt wie ein analoges Theremin funktioniert und welche Komponenten ein Theremin ausmachen. In der zweiten Hälfte werden digitale Lösungsansätze besprochen. Anschliessend wird im Kapitel \ref{sec:Realisierung} Realisierung beschrieben wie die Komponenten realisiert wurden. Im Kapitel \ref{sec:Validierung} Validierung wird als erstes auf die Inbetriebnahme des Antennenoszillators eingegangen. Als nächstes werden die Simulationen des VHDL Codes erläutert. Im letzten Abschnitt wird auf die Inbetriebnahme des VHDL Codes auf dem DE1-SoC Board Bezug genommen.





